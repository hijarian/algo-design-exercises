% chktex-file 19
\documentclass{article}
\usepackage[utf8]{inputenc}
\usepackage[T2A]{fontenc}
\usepackage[russian]{babel}

\usepackage{amsmath}
\usepackage{tikz}
        
\begin{document}

\section{Глава 1}

\subsection{Поиск контрпримеров}

\subsubsection{\#1}

\textbf{Условие.}
Докажите, что значение \(a + b\) может быть меньшим, чем значение \(\min(a, b)\).


Допустим:
\begin{equation}
    a = 0, b=-1
\end{equation}

Тогда:
\begin{eqnarray}
    0 + (-1) = &-1& \\
    &-1& < 0 = \min(0, 1)
\end{eqnarray}

Что и требовалось доказать.

\subsubsection{\#2}

\textbf{Условие.}
Докажите, что значение \(a \times b\) может быть меньшим, чем значение \(\min(a, b)\).

Допустим:
\begin{equation}
    a = \frac{1}{2}, b = \frac{1}{4}
\end{equation}

Тогда:
\begin{eqnarray}
    \frac{1}{2} \times \frac{1}{4} =  \frac{1}{8} & \\
    & \frac{1}{8} < \min(\frac{1}{2}, \frac{1}{4}) = \frac{1}{4}
\end{eqnarray}

Что и требовалось доказать.


\subsubsection{\#3}

\textbf{Условие.}
Начертите сеть дорог с двумя точками \(a\) и \(b\), такими, что маршрут между ними,
преодолеваемый за кратчайшее время, не является самым коротким.

\begin{tikzpicture}
    \node (a)     [circle, draw] {a};
    \node (b)     [right of=a] [circle, draw] {b};
    \node (node1) [below of=a] [circle, draw] {1};
    \node (node2) [below of=b] [circle, draw] {2};
    \draw (a) to node [auto,swap]{10} (b);
    \draw (a) to node [auto,swap]{1} (node1);
    \draw (node1) to node [auto,swap]{1} (node2);
    \draw (node2) to node [auto,swap]{1} (b);
\end{tikzpicture}

Цена маршрута <<a,1,2,b>> равна 4, а цена маршрута <<a,b>> равна 10.
То есть, маршрут, проходящий через больше точек, более длинный, преодолевается за меньшее время.
Что и требовалось доказать.

\subsubsection{\#4}

\textbf{Условие.}
Начертите сеть дорог с двумя точками \(a\) и \(b\),
самый короткий маршрут между которыми не является маршрутом с наименьшим числом поворотов.

\begin{tikzpicture}
    \node (a) [circle,draw] {a};
    \node (k) [below of=a] [circle,draw] {k};
    \node (l) [below of=k] [circle,draw] {l};
    \node (m) [below of=l] [circle,draw] {m};
    \node (n) [right of=a] [circle,draw] {n};
    \node (o) [below of=n, right of=b] [circle,draw] {o};
    \node (b) [below of=m]  [circle,draw] {b};
    \draw (a) -- (n);
    \draw (a) -- (k);
    \draw (k) -- (l);
    \draw (l) -- (m);
    \draw (m) -- (b);
    \draw (b) -- (o);
    \draw (o) -- (n);
\end{tikzpicture}

В данной сети дорог кратчайший маршрут от \(a\) до \(b\) это \(anob\), в котором 3 ребра,
но два поворота.
Однако, маршрут без поворотов вообще, \(aklmb\) не самый короткий --- в нём 4 ребра.
Что и требовалось доказать.

\subsubsection{\#5}

\textbf{Условие.}
Задача о рюкзаке: имея множество целых чисел \(S = \{s_1, s_2, \ldots, s_n \}\) и целевое число \(T\),
найти такое подмножество множества \(S\), сумма которого в точности равна \(T\).
Например, множество \(S = \{1, 2, 5, 9, 10\}\), содержит такое подмножество, сумма элементов которого равна T = 22,
но не T = 23.

Найти контрпримеры для каждого из следующих алгоритмов решения задачи о рюкзаке,
т. е., нужно найти такое множество S и число T, при которых подмножество,
выбранное с помощью данного алгоритма, не до конца заполняет рюкзак,
хотя правильное решение и существует:

\begin{itemize}
    \item вкладывать элементы множества S в рюкзак в порядке слева направо, если они подходят (т. е., алгоритм <<первый подходящий>>);
    \item вкладывать элементы множества S в рюкзак в порядке от наименьшего до наибольшего
          (т. е., используя алгоритм <<первый лучший>>);
    \item вкладывать элементы множества S в рюкзак в порядке от наибольшего до наименьшего.
\end{itemize}

\textbf{Первый подходящий.}
\(T = 10, S = \{9, 2, 2, 2, 2, 2\}\)

\textbf{Первый лучший.}
\(T = 6, S = \{1, 2, 5, 4\}\)

\textbf{От самого большого.}
\(T = 6, S = \{1, 2, 5, 4\}\)

\subsubsection{\#6}

\textbf{Условие.}
Задача о покрытии множества: имея семейство подмножеств \(S_1, \ldots, S_m\) универсального
множества \(U = \{1, \ldots, n\}\), найдите семейство подмножеств \(T \subset S\) 
наименьшей мощности, чтобы \(\bigcup_{t_i} \in T^{t_i} = U\).
Например, для семейства подмножеств \(S_1 = \{1, 3, 5\}\), \(S_2 = \{2, 4\}\), \(S_3 = \{1, 4\}\),
\(S_5 = \{2,5\}\) покрытием множества будет семейство подмножеств \(S_1\) и \(S_2\).

Приведите контрпример для следующего алгоритма: выбираем самое мощное подмножество для покрытия,
после чего удаляем все его элементы из универсального множества; повторяем добавление подмножества,
содержащего наибольшее количество неохваченных элементов, пока все элементы не будут покрыты.

\begin{align}
    S_1 &= \{1, 3, 5, 7, 9\}\\
    S_2 &= \{1, 2\}\\
    S_3 &= \{3, 4\}\\
    S_4 &= \{5, 6\}\\
    S_5 &= \{7, 8\}\\
    S_6 &= \{9, 10\}
\end{align}

Наилучшее покрытие это \(S_2 \cup S_3 \cup S_4 \cup S_5 \cup S_6\),
но жадный алгоритм вначале выберет \(S_1\) как самое мощное подмножество,
а потом не останется ничего другого, как добавлять все остальные подмножества,
то есть, в итоге алгоритм выберет все исходные подмножества в качестве результата работы, \(T = S\).

\end{document}
